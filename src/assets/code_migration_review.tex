\documentclass[11pt,a4paper]{article}
\usepackage[margin=1in]{geometry}
\usepackage[T1]{fontenc}
\usepackage[utf8]{inputenc}
\usepackage{lmodern}
\usepackage{hyperref}
\usepackage{booktabs}
\usepackage{tabularx}
\usepackage{enumitem}
\hypersetup{
  colorlinks=true,
  linkcolor=blue,
  urlcolor=blue,
  citecolor=blue
}
\title{Literature Review: Code Migration from Rule-Based Translators to LLM-Driven Methods}
\author{}
\date{\today}

\begin{document}
\maketitle

\section{Background and Motivation}
Code migration---translating software from one programming language to another---is driven by language obsolescence, platform modernization, performance, maintainability, and interoperability. Traditional rule-based transpilers were designed to reduce manual rewrites of large codebases, but they demand extensive expertise and maintenance as languages evolve. High-profile migrations (e.g., large COBOL-to-Java efforts) illustrate the cost and complexity, motivating research into more flexible automation.

\section{Rule-Based Code Migration}
\textbf{Approach.} Early source-to-source translators rely on hand-crafted rewrite rules applied to the Abstract Syntax Tree (AST). Rules encode mappings between source and target constructs, yielding predictable, stable output when rules are correct.

\textbf{Limitations.} (i) High development/maintenance cost and dual-language expertise; (ii) limited adaptability to new languages and semantics; (iii) difficulty handling ambiguity and context; (iv) often non-idiomatic output that requires manual cleanup.

\section{Statistical and Early Neural Methods}
\subsection{Statistical Machine Translation (SMT)}
Inspired by natural-language MT, SMT treats code as token sequences and learns translation probabilities from parallel corpora. Scarce parallel code and weak semantic guarantees limited adoption for production migrations.

\subsection{Unsupervised Neural Transcompilation}
\textbf{TransCoder} (Meta AI, 2020) demonstrated that fully unsupervised neural translation trained on monolingual GitHub code (via cross-lingual pretraining, denoising, and back-translation) can outperform rule-based baselines at the function level across C++, Java, and Python. Despite strong results, training was computationally expensive and performance declined beyond function granularity.

\section{LLM-Based Code Migration}
Large Language Models (LLMs) pretrained on massive code+text corpora (e.g., GPT, LLaMA families) enable zero/few-shot translation across many language pairs.

\subsection{Advantages}
\begin{itemize}[noitemsep]
  \item Broad language coverage and flexibility without bespoke rule sets.
  \item Ability to propose idiomatic patterns in the target language and handle tricky constructs (e.g., generators, method overloading) by producing semantically equivalent code.
  \item Automation with reduced developer effort for straightforward use cases.
\end{itemize}

\subsection{Limitations and Failure Modes}
Empirical studies show raw LLM outputs can be inconsistent and error-prone:
\begin{itemize}[noitemsep]
  \item \textbf{Context/window limits:} Long files and multi-module projects exceed context, risking broken dependencies.
  \item \textbf{Bug patterns:} Missing imports/definitions, incorrect loops/conditions, inserted or removed logic, wrong data types, brittle input parsing, and formatting mismatches.
  \item \textbf{Semantic drift:} Code may compile yet deviate from source behavior.
  \item \textbf{Unpredictability:} Nondeterministic outputs complicate testing and versioning.
\end{itemize}
A large-scale ICSE'24 study across five languages reported correct translations ranging roughly from 2\% to 47\% for general/code LLMs on diverse benchmarks, underscoring the need for additional guidance and verification.

\section{LLM+Systems: Guidance and Repair}
Recent work augments LLMs with program analysis and feedback loops to raise reliability:

\paragraph{UniTrans (2024).} Generates tests from the source program, augments prompts with I/O contracts, executes translated code, and iteratively repairs failures. Reported sizable gains in computational and exact-match accuracy across multiple LLMs.

\paragraph{TRANSAGENT (2024).} A multi-agent pipeline with an initial translator, a syntax-error fixer, a code-aligner (block mapping via control-flow), and a semantic-error fixer. Uses execution alignment to localize faulty regions before targeted repair; outperforms prior LLM baselines in effectiveness and efficiency.

\paragraph{Program State Alignment / Instrumentation (2024).} Instruments both source and translated code to collect runtime traces and prompt the LLM to align internal states, not just final outputs. On long-code benchmarks, this substantially improved computational accuracy for several models.

\paragraph{Code Co-Evolution / Edit Translation (Codeditor, FSE'23).} Models cross-language \emph{edits} rather than whole-file translation to propagate bug fixes/features across multilingual codebases (e.g., Java/C\#), reflecting how developers actually work.

\section{Comparative Summary}
\begin{table}[h]
\centering
\renewcommand{\arraystretch}{1.2}
\begin{tabularx}{\textwidth}{@{}p{3.2cm}X X p{3.8cm}@{}}
\toprule
\textbf{Category} & \textbf{Strengths} & \textbf{Limitations} & \textbf{Representative Work} \\
\midrule
Rule-based translators & Predictable, stable output; developer control & High rule cost; poor adaptability; ambiguous constructs hard; often non-idiomatic output & Early AST rewrites; commercial transpilers \\
Unsupervised neural (pre-LLM) & No parallel data needed; beats rule-based on functions & Heavy training cost; function-level focus; syntax/semantic errors remain & TransCoder (2020) \\
LLM-only translation & Broad language coverage; handles complex patterns; low setup & Context limits; bug patterns; semantic drift; nondeterminism & ICSE'24 bug taxonomy study \\
LLM + guidance/repair & Execution/test signals localize and fix errors; big accuracy gains & Infra overhead; gains vary by model/task; long code remains hard & UniTrans (2024), TRANSAGENT (2024), State-Alignment (2024) \\
\bottomrule
\end{tabularx}
\end{table}

\section{Trends and Outlook}
The research trajectory moves from hand-crafted rules to data-driven models and then to \emph{LLM-centered systems} that integrate testing, analysis, and repair. Reliable migration at scale will likely combine:
\begin{itemize}[noitemsep]
  \item Static/dynamic analysis (CFG/DFG reasoning, type inference, trace alignment).
  \item Execution-based validation (unit/property tests, metamorphic relations).
  \item Human-in-the-loop adjudication for tricky semantics and API adaptation.
  \item Project-level planning (module partitioning, dependency mapping, API shims) to overcome context limits.
\end{itemize}

\section*{Key References (informal)}
\begin{enumerate}[leftmargin=*]
  \item Lachaux, Rozière, Chanussot, Lample. \emph{Unsupervised Translation of Programming Languages} (TransCoder), 2020.
  \item Pan et al. \emph{Lost in Translation: A Study of Bugs Introduced by LLMs while Translating Code}, ICSE 2024.
  \item Yang et al. \emph{UniTrans: Enhancing LLM-based Code Translation via Test Generation and Repair}, 2024.
  \item Yuan et al. \emph{TRANSAGENT: An LLM-Based Multi-Agent System for Code Translation}, 2024.
  \item Li et al. \emph{Enhancing LLMs in Long Code Translation through Instrumentation and Program State Alignment}, 2024.
  \item Zhang, Nie, Li, Gligoric. \emph{Multilingual Code Co-Evolution Using Large Language Models} (Codeditor), ESEC/FSE 2023.
\end{enumerate}

\end{document}


